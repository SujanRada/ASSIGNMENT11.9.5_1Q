% \iffalse
\let\negmedspace\undefined
\let\negthickspace\undefined
\documentclass[journal,12pt,twocolumn]{IEEEtran}
\usepackage{cite}
\usepackage{amsmath,amssymb,amsfonts,amsthm}
\usepackage{algorithmic}
\usepackage{graphicx}
\usepackage{textcomp}
\usepackage{xcolor}
\usepackage{txfonts}
\usepackage{listings}
\usepackage{enumitem}
\usepackage{mathtools}
\usepackage{gensymb}
\usepackage{comment}
\usepackage[breaklinks=true]{hyperref}
\usepackage{tkz-euclide} 
\usepackage{listings}
\usepackage{gvv}                                        
\def\inputGnumericTable{}                                 
\usepackage[latin1]{inputenc}                                
\usepackage{color}                                            
\usepackage{array}                                            
\usepackage{longtable}                                       
\usepackage{calc}                                             
\usepackage{multirow}                                         
\usepackage{hhline}                                           
\usepackage{ifthen}                                           
\usepackage{lscape}

\newtheorem{theorem}{Theorem}[section]
\newtheorem{problem}{Problem}
\newtheorem{proposition}{Proposition}[section]
\newtheorem{lemma}{Lemma}[section]
\newtheorem{corollary}[theorem]{Corollary}
\newtheorem{example}{Example}[section]
\newtheorem{definition}[problem]{Definition}
\newcommand{\BEQA}{\begin{eqnarray}}
\newcommand{\EEQA}{\end{eqnarray}}
\newcommand{\define}{\stackrel{\triangle}{=}}
\theoremstyle{remark}
\newtheorem{rem}{Remark}
\begin{document}
\parindent 0px
\bibliographystyle{IEEEtran}
\title{Assignment 11.9.5\_1Q}
\author{EE22BTECH11219 - Rada Sai Sujan$^{}$% <-this % stops a space
}
\maketitle
\newpage
\bigskip
\section*{Question}
Show that the sum of $\brak {m+n}^{th}$ and $\brak {m-n}^{th}$ terms of an $A.P.,$ is equal to twice the $m^{th}$ terms.    \\
\solution

\begin{table}[ht]
    \centering
    \def\arraystretch{1.5}
    \begin{tabular}{|p{2cm}|p{2.5cm}|p{2.3cm}|}
    \hline
    PARAMETER & VALUE & DESCRIPTION  \\ \hline
    $$x\brak0$$ & $$x\brak{0}$$ & First term \\ \hline
    $$d$$ & $$d$$ & common difference \\ \hline
    $$x(n)$$ & $$[x\brak{0}+nd]u\brak n$$ & General term of the series  \\ \hline
  \end{tabular}

    \caption{Parameter Table1}
    \label{tab:10.9.5.1}
\end{table}
For an $AP$,
\begin{align}
    x\brak{n}&=[x\brak{0}+nd]u\brak{n}   \\
    \implies x\brak{m+n}+x\brak{m-n}&=[x\brak{0}+\brak{m+n}d]+[x\brak{0}+\brak{m-n}d] \\
    &=2[x\brak{0}+md]   \\
    \therefore x\brak{m+n}+x\brak{m-n}&=2x\brak{m}
\end{align}

\end{document}
